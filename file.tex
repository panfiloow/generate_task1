\documentclass{article}
\usepackage[utf8]{inputenc}
\documentclass{article}
\usepackage[utf8]{inputenc}
\usepackage[english, russian]{babel}
\usepackage{mathtools}
\usepackage{stackrel}
\topmargin=0pt
\leftmargin=0pt
\oddsidemargin=0pt
\usepackage{natbib}
\usepackage{fancyhdr}
\usepackage{graphics}
\usepackage[left=2cm,right=2cm, top=2cm,bottom=2cm,bindingoffset=0cm]{geometry}

\title{MO}

\begin{document}

\begin{center}
	Методы оптимизации -- 2023.
	\textbf{Выполнил ........................................................... группа ..........
Задание по теме <<Графический метод решения задачи ЛП>>, вариант 1}
\end{center}

\begin{flushleft}
	\textit{    Привести к каноническому виду, построить двойственную задачу следующей задачи линейного программирования $(x_1,x_2 \geq 0)$. Угловым точкам $D$ сопоставить базисные множества.}
\end{flushleft}

\begin{displaymath}
	42x_1+17x_2 \rightarrow min
\end{displaymath}
\begin{displaymath}
	8x_1+23x_2 \leq552
\end{displaymath}
\begin{displaymath}
	50x_1+25x_2 \leq2500
\end{displaymath}
\begin{displaymath}
	30x_1+10x_2 \leq1470
\end{displaymath}
\vspace{55}
\end{document}